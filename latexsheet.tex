\documentclass[10pt,landscape]{article}
\usepackage{multicol}
\usepackage{calc}
\usepackage{ifthen}
\usepackage[landscape]{geometry}
\usepackage{hyperref}

% To make this come out properly in landscape mode, do one of the following
% 1.
%  pdflatex latexsheet.tex
%
% 2.
%  latex latexsheet.tex
%  dvips -P pdf  -t landscape latexsheet.dvi
%  ps2pdf latexsheet.ps


% If you're reading this, be prepared for confusion.  Making this was
% a learning experience for me, and it shows.  Much of the placement
% was hacked in; if you make it better, let me know...


% 2008-04
% Changed page margin code to use the geometry package. Also added code for
% conditional page margins, depending on paper size. Thanks to Uwe Ziegenhagen
% for the suggestions.

% 2006-08
% Made changes based on suggestions from Gene Cooperman. <gene at ccs.neu.edu>


% To Do:
% \listoffigures \listoftables
% \setcounter{secnumdepth}{0}


% This sets page margins to .5 inch if using letter paper, and to 1cm
% if using A4 paper. (This probably isn't strictly necessary.)
% If using another size paper, use default 1cm margins.
\ifthenelse{\lengthtest { \paperwidth = 11in}}
	{ \geometry{top=.5in,left=.5in,right=.5in,bottom=.5in} }
	{\ifthenelse{ \lengthtest{ \paperwidth = 297mm}}
		{\geometry{top=1cm,left=1cm,right=1cm,bottom=1cm} }
		{\geometry{top=1cm,left=1cm,right=1cm,bottom=1cm} }
	}

% Turn off header and footer
\pagestyle{empty}


% Redefine section commands to use less space
\makeatletter
\renewcommand{\section}{\@startsection{section}{1}{0mm}%
                                {-1ex plus -.5ex minus -.2ex}%
                                {0.5ex plus .2ex}%x
                                {\normalfont\large\bfseries}}
\renewcommand{\subsection}{\@startsection{subsection}{2}{0mm}%
                                {-1explus -.5ex minus -.2ex}%
                                {0.5ex plus .2ex}%
                                {\normalfont\normalsize\bfseries}}
\renewcommand{\subsubsection}{\@startsection{subsubsection}{3}{0mm}%
                                {-1ex plus -.5ex minus -.2ex}%
                                {1ex plus .2ex}%
                                {\normalfont\small\bfseries}}
\makeatother

% Define BibTeX command
\def\BibTeX{{\rm B\kern-.05em{\sc i\kern-.025em b}\kern-.08em
    T\kern-.1667em\lower.7ex\hbox{E}\kern-.125emX}}

% Don't print section numbers
\setcounter{secnumdepth}{0}


\setlength{\parindent}{0pt}
\setlength{\parskip}{0pt plus 0.5ex}


% -----------------------------------------------------------------------

\begin{document}

\raggedright
\footnotesize
\begin{multicols}{3}


% multicol parameters
% These lengths are set only within the two main columns
%\setlength{\columnseprule}{0.25pt}
\setlength{\premulticols}{1pt}
\setlength{\postmulticols}{1pt}
\setlength{\multicolsep}{1pt}
\setlength{\columnsep}{2pt}

\begin{center}
  \Large{\textbf{Padrino (0.14.0)\ Cheat Sheet}} \\
\end{center}

\section{Generators}
\begin{tabular}{@{}ll@{}}
\verb!project! & generate a new padrino project \\
\end{tabular}


\subsection{Project}
\texttt{padrino generate project NAME}
\begin{itemize}
  \item \texttt{-t} / \texttt{--tiny} \\
    creates controllers.rb, helpers.rb, mailers.rb instead of directories for them
  \item \texttt{-b} / \texttt{--bundle}} \\
    will install the gems automatically
  \item \texttt{-d} / \texttt{--orm} \\
    define the object-relational mapper. Options: 'activerecord', 'couchrest',
    'datamapper', 'dynamoid', 'minirecord', 'mongoid', 'mongomapper', 'mongomatic', 'ohm', 'ripple',
    'sequel', 'none'
  \item \texttt{-t} / \texttt{--test} \\
    define testing framework. Options: 'bacon', 'cucumber', 'minitest', 'rspec', 'testunit', 'none'
  \item \texttt{-s} / \texttt{--script} \\
    define the JavaScript library. Options: 'dojo', 'extcore', 'jquery', 'mootools', 'prototype', 'none'
  \item \texttt{-e} / \texttt{--renderer} \\
    define the rendering engine. Options: 'erb', 'haml', 'liquid', 'slim', 'none'
  \item \texttt{-a} / \texttt{--adapter} \\
    specify the database configuration for config/database.rb file . Options: 'sqlite', 'mysql',
    'mysql2', 'mysql-gem', 'postgres'.
    It only works if the -d option for the object-relational mapper is given.
\end{itemize}


\subsection{Controller}
\texttt{padrino generate controller NAME}
\begin{itemize}
  \item  \\
    Creates: app/controllers/name.rb, app/views/name, app/helpers/name\_helper.rb
  \item \texttt{-d} / \texttt{--destroy} \\
    remove all generated files for the given controller
  \item \texttt{-l} / \texttt{--layout} \\
    defines the layout for the controller
  \item \texttt{-n} / \texttt{--namespace} \\
    defines the namespace for the new controller
  \item \texttt{--no-helper} \\
    no helper will be created
  \item \texttt{-p} / \texttt{--parent} \\
    defines the parent for the new controller
  \item \texttt{-r} / \texttt{--root} \\
    defines the root destination for the padrino application in which the controller should be
    created; config/boots.rb needs to be there in the given padrino application
  \item \texttt{--fields} \\
    defines the fields of the controller; eg. \texttt{padrino generate controller user get:user post:user put:page}
  \item \texttt{-a} / \texttt{--app} \\
    defines the application destination path for the controller (default is /app); eg. \texttt{padrino generate controller app -a test/app}
  \item \texttt{-f} / \texttt{--provides} \\
    define the default return formats for the controller
\end{itemize}


\section{Start}
\begin{tabular}{@{}ll@{}}
\verb!padrino start! & run the padrino app\\
\end{tabular}
\begin{itemize}
  \item \texttt{-p} \\
    specify the port for the running app
\end{itemize}


\end{document}
\end{verbatim}

\rule{0.3\linewidth}{0.25pt}
\scriptsize

Copyright \copyright\ 2014 Winston Chang

\href{http://www.stdout.org/~winston/latex/}{http://www.stdout.org/$\sim$winston/latex/}


\end{multicols}
\end{document}
